%% start of file `Lebenslauf_Template.tex'.
%% Original-Copyright 2006-2010 Xavier Danaux (xdanaux@gmail.com).
%% Changed by gratiswurst.de
%
%
% This work may be distributed and/or modified under the
% conditions of the LaTeX Project Public License version 1.3c,
% available at http://www.latex-project.org/lppl/.

\documentclass[11pt,a4paper]{moderncv}

% moderncv themes
% \moderncvtheme[blue]{casual}                 % optional argument are 'blue' (default), 'orange', 'red', 'green', 'grey' and 'roman' (for roman fonts, instead of sans serif fonts)
\moderncvtheme[blue]{classic}                % idem

% character encoding
% \usepackage[english]{babel}
\usepackage[utf8]{inputenc}                   % replace by the encoding you are using

% adjust the page margins
\usepackage[scale=0.8]{geometry}
\usepackage{graphicx}
\usepackage{url}
% \usepackage{hyperref}
\usepackage{csquotes}
\setlength{\hintscolumnwidth}{3cm}                      % if you want to change the width of the column with the dates
%\AtBeginDocument{\setlength{\maketitlenamewidth}{6cm}}  % only for the classic theme, if you want to change the width of your name placeholder (to leave more space for your address details
%\AtBeginDocument{\recomputelengths}                     % required when changes are made to page layout lengths


%----------------------------------------------------------------------------------
%            Kontaktdaten
%----------------------------------------------------------------------------------
% VORNAME
\firstname{Helge}
% NACHNAME
\familyname{Eichhorn}
%TITEL (optional, ggf. einfach die Zeile lˆschen!)
\title{Curriculum Vitae}
%ADRESSE  (optional, ggf. einfach die Zeile lˆschen!)
\address{Holzhofallee 22b}{64295 Darmstadt, Germany}
%HANDYNUMMER  (optional, ggf. einfach die Zeile lˆschen!)
\mobile{+49176/81023501}
%FESTNETZNUMMER  (optional, ggf. einfach die Zeile lˆschen!)
% \phone{+496132 / 62947}
%EMAIL-ADRESSE  (optional, ggf. einfach die Zeile lˆschen!)
\email{helge@helgeeichhorn.de}
%HOMEPAGE  (optional, ggf. einfach die Zeile lˆschen!)
\homepage{www.helgeeichhorn.de}
%FOTO  (optional, ggf. einfach die Zeile lˆschen!)
%  64pt = Hˆhe des Bildes, 'picture' = Name des Bildes
%\photo[64pt]{picture}

% to show numerical labels in the bibliography; only useful if you make citations in your resume
% \makeatletter
% \renewcommand*{\bibliographyitemlabel}{\@biblabel{\arabic{enumiv}}}
% \makeatother

%----------------------------------------------------------------------------------
%            Inhalt
%----------------------------------------------------------------------------------
\begin{document}
\maketitle
\section{Personal Information}
\cvline{Date of Birth}{19/12/1986}
\cvline{Place of Birth}{Mainz}
\cvline{Nationality}{German}
% \cvline{Geschlecht}{männlich}
% \cvline{Familienstand}{ledig}

\section{Professional Experience}
\cvline{since 2/2013}{Master's Thesis at the European Space Operations Centre (ESOC)%
\begin{itemize}
    \item Mission Analysis Section:
    \begin{itemize}
        \item Design and implementation of a Fortran framework for the numerical propagation and optimization of launcher ascent trajectories.
    \end{itemize}
\end{itemize}
}
\cvline{11/2012~--~1/2013}{Traineeship at the European Space Operations Centre (ESOC)%
\begin{itemize}
  \item Mission Analysis Section:
      \begin{itemize}
          \item Reconstruction of the Apollo 15 trajectory on behalf of ESAC for the re-evalutation of science data.
      \end{itemize}
  \item Gaia Spacecraft Operations:
        \begin{itemize}
          \item Implementation of a Long-Term Mission Planning Tool intended for Mission Planning System validation and as backup during operations.
        \end{itemize}
  \end{itemize}}
% \cvline{since 2012}{Development and maintenance of an open-source Python astrodynamics library%
% \begin{itemize}
%     \item Plyades (\url{https://github.com/helgee/plyades})
% \end{itemize}}

\section{University Education}
\subsection{Master's Degree in \enquote{Mechanical and Process Engineering}}
\cvline{since 10/2007}{Technische Universität Darmstadt}
\cvline{}{Master's Thesis~--~\emph{Knowledge-Based Simulation Models for the Payload Assessment of Launch Systems} \newline Advisor: Prof. Dr.-Ing. Reiner Anderl}
\cvline{25/5/2011}{Bachelor's Thesis~--~\emph{Image Post-Processing of the Impact of Super-Cooled Drops} \newline Advisor: Prof. Dr.-Ing. Cameron Tropea}
\cvline{8/2008~--~10/2008}{Stay abroad: Kunglia Tekniska Högskolan (KTH) Stockholm, Sweden}
\cvline{2/2008~--~3/2008}{Internship: Werner \& Mertz GmbH, Mainz}

%\section{Beruflicher Werdegang}
%\cvline{7/2006 - 9/2006}{Arbeit suchend}
%\cvline{6/2005 - 7/2006}{B¸rokaufmann \newline bei Immergut GmbH in 12345 Muns}
%\cvline{7/2002 - 7/2005}{Ausbildung zum B¸rokaufmann \newline bei Immergut GmbH in 12345 Muns}

\section{Civilian Service}
\cvline{10/2006~--~6/2007}{Paramedic -- DRK Rettungsdienst Rheinhessen-Nahe gGmbH, Mainz}

\section{Education}
\cvline{8/1997~--~3/2006}{Rabanus-Maurus-Gymnasium Mainz}
% \cvline{8/1993 - 7/1997}{Grundschule Wackernheim}
\cvline{1/2003~--~6/2003}{Stay abroad: Belmont Secondary Highschool Victoria(B.C.), Canada}


\section{Other Experience}

\cvline{10/2009~--~10/2012}{Teaching assistant for MATLAB programming and software development at the Department of Computer Integrated Design (DIK) at TU Darmstadt}
\cvline{6/2006~--~9/2006}{Temporary employment at Mainzer Volksbank eG - Organisation Department}

\section{Additional Qualifications}

\subsection{Language Skills}
\cvline{German}{native speaker}
\cvline{English}{fluent}
\cvline{Swedish}{basic skills}

\subsection{IT Skills}
\cvline{Software development}{Python, MATLAB, Fortran 77/95/2003}
\cvline{OS}{Unix, Windows}
\cvline{CAD}{Siemens NX}
\cvline{other}{LaTeX, Microsoft Office, LibreOffice}
% \cvline{C/C++}{Grundkenntnisse}
% \cvline{Java}{Grundkentnisse}
% \cvline{Fortran}{Grundkentnisse}
% \cvline{Python}{Grundkentnisse}
% \cvline{CAD}{Gute Kenntnisse in Unigraphics NX8}
% \cvline{MS Office}{Gute Kenntnisse}
% \cvline{OpenOffice}{Gute Kenntnisse}
% \cvline{}{Administration von Windows- und Unix-Systemen}
%
%\section{Zertifikate}
%\cvline{24.05.2011 - 25.07.2022}{Jodeldiplom}
%\cvline{22.07.2009}{Selbst- und Zeitmanagement}
%\cvline{11.12.2008}{Projektmanagaement}

%\section{Liste mit irgendwelchen Sachen}
%\renewcommand{\listitemsymbol}{-} % Das Aufz‰hlungszeichen f¸r die folgende Liste
%\cvlistdoubleitem{Das kann ich}{Das hier auch}
%\cvlistdoubleitem{Und sowas ist ganz toll}{Blablabla}
%\cvlistdoubleitem{Und noch was}{}
\strut\\
\strut\\
\strut\\
Darmstadt, \today\\ % Aktuelles Datum und Stadt
%\includegraphics[scale=0.7]{us.jpg}\\ % Unterschrift. Löschen falls nicht vorhanden
\end{document}


%% end of file `Lebenslauf_Template.tex'.
