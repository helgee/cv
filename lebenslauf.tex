%% start of file `Lebenslauf_Template.tex'.
%% Original-Copyright 2006-2010 Xavier Danaux (xdanaux@gmail.com).
%% Changed by gratiswurst.de
%
%
% This work may be distributed and/or modified under the
% conditions of the LaTeX Project Public License version 1.3c,
% available at http://www.latex-project.org/lppl/.

\documentclass[10pt,a4paper]{moderncv}

% moderncv themes
%\moderncvtheme[blue]{casual}                 % optional argument are 'blue' (default), 'orange', 'red', 'green', 'grey' and 'roman' (for roman fonts, instead of sans serif fonts)
\moderncvtheme[blue]{classic}                % idem

% character encoding
\usepackage[english, ngerman]{babel}
\usepackage[utf8]{inputenc}                   % replace by the encoding you are using

% adjust the page margins
\usepackage[scale=0.8]{geometry}
\usepackage{graphicx}
\usepackage{url}
% \usepackage{hyperref}
\usepackage{csquotes}
\setlength{\hintscolumnwidth}{3.1cm}                      % if you want to change the width of the column with the dates
%\AtBeginDocument{\setlength{\maketitlenamewidth}{6cm}}  % only for the classic theme, if you want to change the width of your name placeholder (to leave more space for your address details
%\AtBeginDocument{\recomputelengths}                     % required when changes are made to page layout lengths


%----------------------------------------------------------------------------------
%            Kontaktdaten
%----------------------------------------------------------------------------------
% VORNAME
\firstname{Helge}
% NACHNAME
\familyname{Eichhorn}
%TITEL (optional, ggf. einfach die Zeile lˆschen!)
\title{Lebenslauf}
%ADRESSE  (optional, ggf. einfach die Zeile lˆschen!)
\address{Nelly-Sachs-Str. 23}{64347 Griesheim}
%HANDYNUMMER  (optional, ggf. einfach die Zeile lˆschen!)
\mobile{+49 171/8106299}
%FESTNETZNUMMER  (optional, ggf. einfach die Zeile lˆschen!)
% \phone{+496132 / 62947}
%EMAIL-ADRESSE  (optional, ggf. einfach die Zeile lˆschen!)
\email{helge@helgeeichhorn.de}
%HOMEPAGE  (optional, ggf. einfach die Zeile lˆschen!)
% \homepage{www.helgeeichhorn.de}
%FOTO  (optional, ggf. einfach die Zeile lˆschen!)
%  64pt = Hˆhe des Bildes, 'picture' = Name des Bildes
% \photo[40pt]{ei.jpg}

% to show numerical labels in the bibliography; only useful if you make citations in your resume
% \makeatletter
% \renewcommand*{\bibliographyitemlabel}{\@biblabel{\arabic{enumiv}}}
% \makeatother

%----------------------------------------------------------------------------------
%            Inhalt
%----------------------------------------------------------------------------------
\begin{document}
\sffamily
\maketitle
\section{Persönliche Informationen}
\cvline{Geburtsdatum}{19.12.1986}
% \cvline{Place of Birth}{Mainz}
\cvline{Nationalität}{deutsch}
% \cvline{Geschlecht}{männlich}
% \cvline{Familienstand}{ledig}

\section{Berufserfahrung}
\subsection{Head of Mission Operations}
\cvline{seit 7.2018}{Planetary Transportation Systems GmbH (vormals PTScientists GmbH)%
    \begin{itemize}
        \item Leitung des Missionsanalyse- und Flugdynamikteams
        \item Architekt und leitender Entwickler für das Bodensegment
        \item Technischer Koordinator für die Flugbetriebkooperation mit ESOC
    \end{itemize}
}
\subsection{Software Engineer -- Astrodynamics \& Satellite Simulation}
\cvline{2.2017~--~6.2018}{Telespazio VEGA Deutschland GmbH%
    \begin{itemize}
        \item Entwicklung der Lagekontrolle für das Missionsplanungssystem der ASI PRISMA Mission
        \item Weiterentwicklung und Wartung der UMF (Universal Modelling Framework) Komponente des SIMULUS Satellitensimulationssystems der ESA
        \item Validierung des EGS-CC Monitoring and Command Model Kernels
    \end{itemize}
}
\subsection{Freiberuflicher Berater in der Missionsanalyse -- Nebentätigkeit}
\cvline{1.2014~--~10.2016}{ESOC -- European Space Operations Centre, European Space Agency%
    \begin{itemize}
        \item Design und Entwicklung einer modernen Softwareinfrastruktur für Mondmissionsanalyse
        \item Weiterentwicklung und Wartung der Aufstiegsbahnoptimierungssoftware FASTOP
        \item Evaluation von Programmiersprachen für die nächste Generation von Astrodynamiksoftwaresystemen
    \end{itemize}
}
\subsection{Wissenschaftlicher Mitarbeiter am Fachgebiet für Datenverarbeitung in der Konstruktion (DiK)}
\cvline{11.2013~--~10.2016}{Technische Universität Darmstadt%
            \begin{itemize}
                \item Forschung im Bereich Industrie 4.0, Datenmodellierung und technische Informatik
                \item Industriekooperation mit der Airbus Group (2015--2016).
                \begin{itemize}
                    \item Evaluation von LOTAR-konformen Produktdatenarchivsystemen
                \end{itemize}
                \item DFG-gefördertes Forschungsprojekt \emph{SCoPE -- Smart Components within Smart Production Processes and Environments} (2014--2015)
                \begin{itemize}
                    \item Entwicklung eines integrierten Bauteildatenmodells für die Industrie 4.0
                \end{itemize}
                \item Unterstützung der Vorlesung \emph{Virtuelle Produktentwicklung C -- Produkt- und Prozessmodellierung}
            \end{itemize}
}
\subsection{Praktikum und Masterarbeit in der Missionsanalyse}
\cvline{11.2012~--~10.2013}{ESOC -- European Space Operations Centre, European Space Agency%
\begin{itemize}
    \item Design und Entwicklung der Aufstiegsbahnoptimierungssoftware FASTOP
    \item Rekonstruktion der Apollo-15-Flugbahn für die Reevaluierung von Röntgenstrahlungsmessungen
\end{itemize}
}

\section{Akademische Laufbahn}
\subsection{Promotion im Maschinenbau}
\cvline{seit 11.2013}{Technische Universität Darmstadt}
\cvline{erwartet 2020}{Dissertation~--~\emph{Model-Based Space Mission Design (Arbeitstitel)} \newline Betreuer: Prof. Dr.-Ing. Reiner Anderl}
\subsection{Master of Science in \enquote{Mechanical and Process Engineering}}
\cvline{10.2011~--~10.2013}{Technische Universität Darmstadt}
\cvline{21.10.2013}{Masterthesis~--~\emph{Knowledge-Based Simulation Models for the Payload Assessment of Launch Systems} \newline Betreuer: Prof. Dr.-Ing. Reiner Anderl}
\subsection{Bachelor of Science in \enquote{Mechanical and Process Engineering}}
\cvline{10.2007~--~9.2011}{Technische Universität Darmstadt}
\cvline{25.5.2011}{Bachelorthesis~--~\emph{Image Post-Processing of the Impact of Super-Cooled Drops} \newline Betreuer: Prof. Dr.-Ing. Cameron Tropea}
\cvline{8.2008~--~10.2008}{Auslandsaufenthalt: Kunglia Tekniska Högskolan (KTH) Stockholm, Schweden}
% \cvline{2/2008~--~3/2008}{Internship: Werner \& Mertz GmbH, Mainz}

\section{Zivildienst}
\cvline{10.2006~--~6.2007}{Rettungssanitäter -- DRK Rettungsdienst Rheinhessen-Nahe gGmbH, Mainz}

\section{Schulbildung}
\cvline{8.1997~--~3.2006}{Rabanus-Maurus-Gymnasium, Mainz}
% \cvline{8/1993 - 7/1997}{Grundschule Wackernheim}
\cvline{1.2003~--~6.2003}{Auslandsaufenthalt: Belmont Secondary Highschool Victoria, B.C., Canada}


% \section{Other Experience}
%
% \cvline{10/2009~--~10/2012}{Teaching assistant for MATLAB programming and software development at Department of Computer Integrated Design (DIK) -- TU Darmstadt}
% \cvline{6/2006~--~9/2006}{Temporary employment at Mainzer Volksbank eG - Organisation Department}

\section{Zusätzliche Qualifikationen}

\subsection{Sprachkentnisse}
\cvline{Deutsch}{Muttersprache}
\cvline{Englisch}{Verhandlungssicher}

\subsection{Softwareentwicklung}
\cvline{Experte}{Julia, Python, MATLAB, Fortran}
\cvline{Fortgeschritten}{Java, C/C++, Bash}
\subsection{IT-Kentnisse}
\cvline{OS}{Linux, Windows, macOS}
% \cvline{CAD}{Siemens NX}
\cvline{Büroanwendungen}{LaTeX, Microsoft Office, LibreOffice}
% \cvline{C/C++}{Grundkenntnisse}
% \cvline{Java}{Grundkentnisse}
% \cvline{Fortran}{Grundkentnisse}
% \cvline{Python}{Grundkentnisse}
% \cvline{CAD}{Gute Kenntnisse in Unigraphics NX8}
% \cvline{MS Office}{Gute Kenntnisse}
% \cvline{OpenOffice}{Gute Kenntnisse}
% \cvline{}{Administration von Windows- und Unix-Systemen}
%
%\section{Zertifikate}
%\cvline{24.05.2011 - 25.07.2022}{Jodeldiplom}
%\cvline{22.07.2009}{Selbst- und Zeitmanagement}
%\cvline{11.12.2008}{Projektmanagaement}

%\section{Liste mit irgendwelchen Sachen}
%\renewcommand{\listitemsymbol}{-} % Das Aufz‰hlungszeichen f¸r die folgende Liste
%\cvlistdoubleitem{Das kann ich}{Das hier auch}
%\cvlistdoubleitem{Und sowas ist ganz toll}{Blablabla}
%\cvlistdoubleitem{Und noch was}{}
\section{Relevante Publikationen}
\begin{itemize}
    \item Eichhorn, Helge; Steindorf, Lukas; Cano, Juan Luis: Astrodynamics.jl: A Julia-Based Open Source Framework for Orbital Mechanics. In: Proceedings of the 7th International Conference on Astrodynamics Tools and Techniques (ICATT 2018), Oberpfaffenhofen, Germany, 6-9 November, 2018.
    \item Eichhorn, Helge; Cano, Juan Luis; McLean, Frazer; Anderl, Reiner: A Comparative Study of Programming Languages for Next-Generation Astrodynamics Systems. In: CEAS Space Journal (2017). \url{https://doi.org/10.1007/s12567-017-0170-8}.
    \item Eichhorn, Helge; Anderl, Reiner: Plyades: A Python Library for Space Mission Design. In: Proceedings of the 8th European Conference on Python in Science (EuroSciPy 2015), Cambridge, United Kingdom, 28-29 August, 2015 (pp. 9-12).
\end{itemize}
\bigskip
\bigskip
\bigskip
\bigskip
\bigskip
Griesheim, \today % Aktuelles Datum und Stadt
%\includegraphics[scale=0.7]{us.jpg}\\ % Unterschrift. Löschen falls nicht vorhanden
\end{document}


%% end of file `Lebenslauf_Template.tex'.
